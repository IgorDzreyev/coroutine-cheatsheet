% !TeX Options = -shell-escape

\documentclass[twoside,twocolumn, 10pt]{article}
\usepackage[english]{babel}
\usepackage{fancyhdr}
\usepackage{minted}
\usepackage{xcolor}

\setminted[c++]
{
fontsize=\footnotesize,
autogobble=true
}

\author{Dawid Pilarski}
\date{}
\title{Coroutines cheatsheet}

\begin{document}

\section{Coroutine body}

Every time, the compiler encounters one of the following
keywords:
\begin{itemize}
	\item co\_return
	\item co\_yield
	\item co\_await
\end{itemize}

function in which body the keyword was encountered is transformed into
the coroutine. This is performed in the following schema:

\inputminted{c++}{code-examples/theory-custom-coroutine/coroutine-body.cpp}

\section{promise\_type}

Promise\_type is the type used by the compiler to control behavior of the coroutine.
It should be defined (as publicly available) either as a member of the coroutine type like:

\begin{minted}{c++}
returned_type::promise_type
\end{minted}

or as a member of the specialization of the coroutine\_traits:

\begin{minted}{c++}
namespace std{
	template <>
  struct coroutine_traits<returned_type>{
    struct promise_type{
      //definition
    };
  };
}
\end{minted}

The functions, that steer the coroutine behavior are listed below:

\inputminted{c++}{code-examples/theory-custom-coroutine/promise-type.hpp}

\section{co\_yield}
	Each time, when compiler sees co\_yield keyword, the following
	code is generated:

\begin{minted}{c++}
	co_await promise.yield_value(<expression>);
\end{minted}

	If your type is not meant to support yield
	expression, you can suppress it with:

\begin{minted}{c++}
  auto yield_value() = delete;
\end{minted}

  in the definition of your promise\_type.

\section{co\_return}
	To finish the coroutine and optionally return the value from it
	one can use the co\_return expression. Such expression is also
	translated by the compiler depending on the operand of the
	co\_return keyword.

	Void expressions and co\_return without any expression is
	translated into following form:
	\begin{minted}{c++}
		<optional_expression>;
		promise.return_void();
	\end{minted}

	while for the non-void expressions, following code is generated:

	\begin{minted}{c++}
		promise.return_value(<expression>);
	\end{minted}

\section{coroutine\_handle}

	Coroutine handle is our object, that directly operates on the coroutine
	(it can for example resume it or delete it). It's API is as follows:

	\inputminted{c++}{code-examples/theory-custom-coroutine/coroutine-handle.hpp}

	\section{Awaitable primitives}

	The standard library defined two primitives, that can be
	operands of the co\_await operator, namely:
	\begin{itemize}
		\item std::suspend\_always - causes suspension of the coroutine
		\item std::suspend\_never - is a no-op
	\end{itemize}

	\section{Creating awaiter}

	The co\_await operator needs so called awaiter object to know
	how should a coroutine behave on awaiting an awaitable object.

	The awaiter object is created in following way:

	\begin{itemize}
		\item The await\_transform function form the
		 promise\_type is executed on the co\_await operand,
		\item co\_await operator is searched in the body of the awaitable,
		\item if not found global co\_await operator is searched for,
		\item if not found awaitable becomes the awaiter
	\end{itemize}

	\section{Awaiter}

	Awaiter object must have following functions defined in it's body:

	\inputminted{c++}{code-examples/theory-custom-coroutine/awaiter.hpp}

	Their responsibility:
	\begin{itemize}
		\item await\_ready - knows whether the awaitable is finished and result can be fetched from it,
		\item await\_suspend - knows how to await on the awaitable (usually how to resume it),
		\item await\_resume - result of this function evaluation is the result of the whole co\_await expression.
	\end{itemize}

	\section{co\_await transformation}

	Whenever co\_await keyword is encountered by compiler, compiler generates following code (besides the procedure for acquiring awaiter)

	\inputminted{c++}{code-examples/theory-custom-coroutine/co-await-transformation.cpp}

	where await\_suspend expression is one of the following:

	\begin{itemize}
		\item when await\_suspend returns void \\ \inputminted[firstline=2, lastline=8]{c++}{code-examples/theory-custom-coroutine/co-await-transformation-suspend.cpp}
		\item when await\_suspemd returns bool \\ \inputminted[firstline=11, lastline=19]{c++}{code-examples/theory-custom-coroutine/co-await-transformation-suspend.cpp}
		\item when await\_suspemd returns another coroutine\_handle \\ \inputminted[firstline=22, lastline=31]{c++}{code-examples/theory-custom-coroutine/co-await-transformation-suspend.cpp}
	
	\end{itemize}

\end{document}